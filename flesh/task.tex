\chapter{Постановка задачи}
В рамках данной работы необходимо разработать схему геймификации занятий бегом, спроектировать архитектуру мобильного приложения для платформы Android, реализовать работу с сервисами геоданных, определить формат данных и представить обработку интерактивных сценариев в этом формате, а так же провести тестирование разработанного приложения.

Под геймификацией~\autocite{gamification} здесь и далее будем понимать внедрение игровых форм взаимодействия в неигровой контекст: работу, учебу и повседневную жизнь.

Разработка данного приложения полностью покрывает требования к компетенциям, получаемым в рамках текущего направления подготовки: элементы геймификации покрывают собой игровую часть, тогда как само приложение в целом является мобильным и соответствует части разработки мобильных приложений.

В рамках постановки задачи по разработке приложения была проведена некоторая подготовительная работа, включающая в себя следующие пункты:
\begin{itemize}
	\item Определение технического стека разработки.
	\item Рассмотрение возможных технических рисков.
	\item Определение базовой функциональности приложения.
	\item Определение методов геймификации.
\end{itemize}

Относительно технического стека разработки было принято решение создавать мобильное приложение под платформу Android. Подробнее об этом расписано в главе \autoref{chap:stack}.


Каждый из возможных технических рисков рисков является отдельным пунктом в списке базовой функциональности игры. Блок функциональности был назван «риском» потому, что при его реализации могут возникнуть непредвиденные трудности, влекущие за собой разного рода риски.

Ниже представлены базовые блоки функционала:
\begin{enumerate}
	\item Обработка геолокации.
	\item Управление воспроизведением аудио.
	\item Движение игрока по сюжетной линии.
	\item Ядро управления геймификацией активности.
	\item Базовый пользовательский интерфейс.
	\item Отслеживание и ограничение длительности игровой сессии.
\end{enumerate}

Схема геймификации должна включать в себя следующее:
\begin{itemize}
	\item Интерактивный сценарий активности
	\item Безусловная мотивация во время активности
	\item Создание персональных условий соревновательности с системой
	\item Позитивная мотивация по достижении заданных условий
\end{itemize}