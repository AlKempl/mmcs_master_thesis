\chapter{Программная реализация сбора биометрии}
В качестве дополнительного метода защиты здоровья пользователя было принято решение анализировать его пульс во время тренировок.
Получение пульса возможно путём взаимодействия с устройствами сбора пользовательской биометрии, такими как умные часы и браслеты.
Для определения доступных возможностей был проведён сравнительный анализ нескольких методов получение биометрических данных:
\begin{itemize}
	\item Выполнение запросов к датчикам биометрии напрямую на носимом устройстве.
	\item Выполнение запроса к сервисам биометрии Google.
	\item Выполнение запроса к агрегатору биометрии на устройстве от Google.
	\item Выполнение запроса к агрегатору биометрии вендора носимых устройств.
\end{itemize}
\smallskip
Результат сравнительного анализа представлен в таблице \ref{tab:biometrics_comparison}.

\newcolumntype{b}{X}
\newcolumntype{s}{>{\hsize=.4\hsize}X}
\newcolumntype{t}{>{\hsize=.25\hsize}X}

\makesavenoteenv{tabular}
\makesavenoteenv{table}

\begin{table}[H]
	\centering
	\caption{\label{tab:biometrics_comparison}Результаты сравнительного анализа методов получения биометрии}
	\small
	\begin{threeparttable}
	\begin{tabularx}{\textwidth}{sbt}
		\toprule
		Метод сбора \newline биометрии
		 & Описание и ограничения
		 & Вывод \\
		\midrule\midrule
		Прямой запрос к датчикам носимого устройства
		  & Предоставляет полные\tnote{1} данные в реальном времени\tnote{2}. \newline Для доступа к датчикам носимого устройства требуется сборка приложения-компаньона для загрузки на носимое устройство
		   & Не\newline подходит \\\midrule
		Запрос к сервисам биометрии Google
		& Предоставляет полные исторические\tnote{3} данные. \newline Для доступа к биометрии требуется платный аккаунт, с которого отправляются запросы к API Google, подразумевается постоянное наличие соединения с Интернетом.
		& Не\newline подходит \\\midrule
		Запрос к агрегатору биометрии на устройстве от Google
		& Предоставляет полные\tnote{1} исторические\tnote{3} данные. \newline Реализуется через запущенный в тестирование в 2022 г. проект Google Health SDK. Требуется установка приложения-компаньона на устройство, настройка синхронизации и предоставление разрешений пользователем. Представляется неинтуитивным и требует много дополнительных действий со стороны конечного пользователя.
		& Не\newline подходит \\\midrule
		Запрос к агрегатору биометрии вендора носимых устройств
		& Предоставляет полные\tnote{1} исторические\tnote{3} данные. \newline Для доступа к данным требуется подключение к приложению пакета SDK соответствующего вендора.
		& Подходит \\
		\bottomrule
	\end{tabularx}
\smallskip
\footnotesize
\begin{tablenotes}
	\item[1] Без ограничений на объем получаемых данных и с обширным списком доступных биометрических показателей.
	\item[2] Предоставляется возможность отслеживать изменения значений показателя в реальном времени и реагировать на них.
	\item[3] Предоставляется возможность выгрузить историю значений показателя за некоторый временной интервал в прошлом.
\end{tablenotes}
\end{threeparttable}
\end{table}
\smallskip

Исходя из результатов сравнительного анализа методов получения биометрических данных, было принято решение использовать запросы к агрегатору данных конкретного вендора носимых устройств.
С учетом особ