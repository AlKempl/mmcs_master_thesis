\chapter{Анализ результатов тестирования приложения}
\section{Методы тестирования Android приложений}
Для обеспечения качества написанного кода и удобства использования готового программного продукта принято на всех стадиях разработки проводить тестирование приложения. 

Традиционно выделяется несколько способов тестирования Android приложений:
\begin{itemize}
	\item Автоматическое тестирование
	\item Ручное тестирование
\end{itemize}

Автоматическое тестирование в свою очередь подразделяется на следующие методы:
\begin{itemize}
	\item Автотесты для проверки кодовой базы
	\item Автотесты для проверки работы пользовательского интерфейса
\end{itemize}
В рамках данной работы был применён вариант автоматического тестирования кодовой базы. Способы его применения будут рассмотрены в разделе \ref{sec:autotesting}.

Ручное тестирование обычно состоит из следующих этапов:
\begin{enumerate}
	\item Тестирование командой продукта
	\item Закрытое тестирование отобранной группой пользователей
	\item Открытое тестирование всеми желающими с выкладыванием приложения в открытый доступ
\end{enumerate}
Для данной работы были проведены первые два этапа, ход которых представлен в разделе \ref{sec:usertesting}.

\section{Автоматическое тестирование}
\label{sec:autotesting}
Для автоматического тестирования кодовой базы приложения была использована стандартная библиотека Junit4. Она позволяют создавать методы для автоматического локального юнит-тестирования созданного кода приложения. Такие тесты выполняются в локальной виртуальной машине Java на устройстве тестирования.

Для тестирования кода в Android существует надстройка над JUnit4 \textemdash\space AndroidJunit4. Она позволяет создавать так называемые ``Инструментированные'' тесты. Их преимущество заключается в том, что они запускаются в схеме собираемого приложения и имеют доступ к его активностям, репозиториям и базовому контексту приложения.

Для тестирования приложения был выбран вариант инструментированных тестов.

Каждый тестовый метод должен содержать аннотацию $@RunWith(AndroidJUnit4::class)$ для предоставления доступа к Instrumentation API.

Примеры методов автоматического тестирования методов загрузки сценария приведены в приложении \ref{list_sec:autotests}.

\section{Закрытое пользовательское тестирование}
\label{sec:usertesting}
Закрытое пользовательское тестирование позволяет проверить стабильность работы приложения и удобство его использования в реальных боевых условиях:
\begin{enumerate}
	\item Производится выборка пользователей, заинтересованных в тестировании
	\item Проводится вводный инструктаж пользователей по следующим пунктам:
	\begin{enumerate}
		\item запуск приложения;
		\item использование приложение;
		\item тестовые сценарии использования приложения, которые надо пройти;
		\item действия в нештатных ситуациях;
		\item правила сообщения о найденных ошибках.
	\end{enumerate}
	\item Объявляется начало тестирования с рассылкой приложения.
	\item Производится сбор результатов тестирования.
\end{enumerate}

Для тестирования данного приложения были привлечены двое пользователей, проинструктированы и приглашены к прохождению тестовых сценариев.
Пользовательское тестирования проводилось в два этапа: одно на раннем и второе позднем этапах разработки приложения.

Результаты пользовательского тестирования описаны в следующем разделе.

\section{Анализ результатов проведённого тестирования}
\subsection*{Первый этап}
Блабла

\subsection*{Второй этап}
Блабла