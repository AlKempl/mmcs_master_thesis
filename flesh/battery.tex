\chapter{Оптимизация использования ресурсов устройства}
Приложение, работающее в фоне и производящее постоянные запросы определенно имеет большой потенциал для оптимизации по использованию ресурсов устройства. Далее приведены некоторые рассуждения относительно влияния постоянной работы приложения на состояние устройства и возможные варианты доработок приложения для улучшения пользовательского опыта и снижения нагрузки на устройство.

\section{Влияние частоты измерения геолокации на расход заряда аккумулятора устройства}
\subsection*{Аспекты, влияющие на расход заряда аккумулятора}
В первую очередь следует определить три аспекта~\autocite{battery}, с которыми напрямую связан повышенный расход заряда аккумулятора устройства:
\begin{description}
	\item[Точность определения геолокации] В общем случае – чем выше точность, тем больше потребление заряда.
	\item[Частота запросов данных датчиков геолокации] Чем чаще происходит опрос датчиков и их сообщение с источниками данных (спутники, сетевые устройства, сотовые вышки, и т.д.), тем выше расход заряда.
	\item[Время ожидания отклика сервиса геолокации] Меньшая скорость обычно требует меньше ресурсов аккумулятора.
\end{description}

\subsubsection*{Точность определения геолокации}
Этим параметром можно управлять с помощью специального метода setPriority, который принимает в качестве аргумента следующие значения:
\begin{description}
	\item[PRIORITY\_HIGH\_ACCURACY] в этом режиме приложение будет получать наиболее точные данные о геолокации устройства на основе всех возможных источников: GPS, WiFi, вышки сотовой связи и множество доступных на устройстве датчиков.
	\item[PRIORITY\_BALANCED\_POWER\_ACCURACY] этот режим представляет приложению достаточно точные данные о геолокации устройства, в щадящем для аккумулятора режиме: обращения к GPS производятся максимально редко, обычно используется комбинация запросов к WiFi и вышкам сотовой связи.
	\item[PRIORITY\_LOW\_POWER] данное значение точности задает более строгий режим экономии электроэнергии устройства: сервис геолокации опирается исключительно на данные вышек сотовой связи, не используя никакие другие источники. Такой подход дает грубое приближение к реальной геолокации устройства с точностью до части населенного пункта, при этом затрачивая минимум заряда аккумулятора.
	\item[PRIORITY\_NO\_POWER] в этом режиме данные геолокации переиспользуются из результатов запросов геолокации другими приложениями, для которых они уже были получены.
\end{description}


\subsubsection*{Частота запросов данных геолокации}
Этим параметром можно управлять с помощью следующих методов:
\begin{description}
	\item[setInterval] задает интервал за который геолокация будет запрашиваться конкретно для этого приложения;
	\item[setFastestInterval] задает интервал за который геолокация будет доставляться этому приложению с результатам запросов другими приложениями.
\end{description}
\smallskip
Хорошей практикой является указание максимально возможного значения первым методом, особенно для сбора геолокации в фоновом режиме.


\subsubsection*{Время ожидания отклика сервиса геолокации}
Значение этого атрибута задается с помощью метода setMaxWaitTime и, как правило, в несколько раз больше, чем значение, указанное в setInterval.
Этот атрибут откладывает доставку результатов определения геолокации, в таком случае за один раз может быть доставлен батч из нескольких результатов с изменениями геолокации.


\subsection*{Способы экономии заряда}
Учитывая специфику приложения, можно отметить следующие факторы, важные для определения стратегии определения методов оптимизации:
\begin{itemize}
	\item Крайне нежелательно откладывать доставку результатов с данными геолокации, так как данные нужны в режиме реального времени.
	\item Физическая активность в открытом пространстве с элементами геймификации может быть потенциально травмоопасной и для минимизации рисков взаимодействия с объектами окружающей инфраструктуры, действительно несущими риск нанесения травм пользователю, требуется знать наиболее точное местоположение пользователя с его устройством.
\end{itemize}
\smallskip
Исходя из всего вышеперечисленного можно предпринять следующие попытки по оптимизации расхода заряда устройства:
\begin{enumerate}
	\item Установить точность на уровне не ниже PRIORITY\_BALANCED\_POWER\_ACCURACY.
	\item Установить частоту запросов к сервисам геолокации не ниже раза в 20 секунд.
	\item Установить время ожидания доставки обновлений не менее 5 секунд.
\end{enumerate}
\smallskip
Эти значения предположительно принесут оптимизацию расхода батареи работой приложения примерно в 3 раза (учитывая текущие значения в минимально рабочем прототипе приложения).


\section{Влияние работы приложения на используемую память устройства}
На основе схемы базы имеется приблизительный~\autocite{sqlite_datatypes} объем данных в одной строке, представляющей собой один объект класса LocationEntity – 90 байт. При получении до 100 объектов в секунду приложение производит 900 байтов. Так приложение займет объем в 1 Гб за примерно 300 часов непрерывной работы. Совершенно очевидно, что такой объем данных, создаваемых и хранимых приложением, не нужен ни приложению, ни пользователю устройства. 
В качестве возможности избежать такой нерациональной траты памяти предлагается:
\begin{enumerate}
	\item Предусмотреть процессы периодической очистки сохранённых данных, возможно, оставляя данные за последние $N$ часов, где $N$ строго меньше $150$.
	\item Рассмотреть возможность сохранять не все пришедшие в батче данные об изменениях локации, а только часть в $M/K$, где $K$ \textemdash\space количество изменений, пришедших в батче, а $M$ \textemdash\space количество записываемых из них изменений такое, что $1 \le M \le [K/2]$.
\end{enumerate}
\smallskip
Такие меры сохранения памяти устройства определённо повысят качество пользовательского опыта и снизят влияние от использования разрабатываемого приложения на другие приложения потенциального пользователя и его устройство в целом.
