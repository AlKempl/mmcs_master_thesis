\chapter{Листинги}

\section{Автоматические тесты кодовой базы}
\label{list_sec:autotests}
\begin{ListingEnv}
	\caption{Тестирование успешной предобработки сценария геймификации}
	\UseRawInputEncoding
	\footnotesize
	\lstinputlisting[language=Kotlin]{flesh/listings/scenario_parse_test.kt}
	%\begin{lstlisting}[language=Kotlin]
	%\input{flesh/listings/GeofenceBroadcastReceiver}
	%\end{lstlisting}
	\label{list:scenario_parse_test}
\end{ListingEnv}

\begin{ListingEnv}
	\caption{Тестирование предобработки конкретного сценария}
	\UseRawInputEncoding
	\footnotesize
	\lstinputlisting[language=Kotlin]{flesh/listings/scenario_mmcs_test.kt}
	%\begin{lstlisting}[language=Kotlin]
	%\input{flesh/listings/GeofenceBroadcastReceiver}
	%\end{lstlisting}
	\label{list:scenario_mmcs_test}
\end{ListingEnv}

\begin{ListingEnv}
	\caption{Проверка корректной обработки полиморфных объектов сценария}
	\UseRawInputEncoding
	\footnotesize
	\lstinputlisting[language=Kotlin]{flesh/listings/scenario_poly_test.kt}
	%\begin{lstlisting}[language=Kotlin]
	%\input{flesh/listings/GeofenceBroadcastReceiver}
	%\end{lstlisting}
	\label{list:scenario_poly_test}
\end{ListingEnv}

\section{Широковещательные сообщения о точке интереса на карте}
\label{list_sec:broadcasting}
\begin{ListingEnv}
	\caption{Действие, вызываемое на вход в отслеживаемую точку интереса на карте}
	\UseRawInputEncoding
	\footnotesize
	\lstinputlisting[language=Kotlin]{flesh/listings/GeofenceBroadcastReceiver_OnEntered.kt}
	\label{list:geofence_brr_on_entered}
\end{ListingEnv}

\begin{ListingEnv}
	\caption{Код обработки полученного широковещательного сообщения о точке интереса на карте}
	\UseRawInputEncoding
	\footnotesize
	\lstinputlisting[language=Kotlin]{flesh/listings/GeofenceBroadcastReceiver_OnReceive.kt}
	\label{list:geofence_brr_on_receive}
\end{ListingEnv}

\section{Генерация таймеров для виртуальных препятствий}
\begin{ListingEnv}
	\caption{Генерация таймера для инициализации препятствия}
	\UseRawInputEncoding
	\footnotesize
	\lstinputlisting[language=Kotlin]{flesh/listings/generate_obstacle_start_timer.kt}
	\label{list:generate_obstacle_start_timer}
\end{ListingEnv}

\begin{ListingEnv}
	\caption{Генерация таймера для валидации активного препятствия}
	\UseRawInputEncoding
	\footnotesize
	\lstinputlisting[language=Kotlin]{flesh/listings/generate_obstacle_finish_timer.kt}
	\label{list:generate_obstacle_finish_timer}
\end{ListingEnv}

\newpage
\section{Работа с Samsung Health SDK}
\label{list_sec:shm}
\begin{ListingEnv}
	\caption{Проверка требуемых разрешений}
	\UseRawInputEncoding
	\footnotesize
	\lstinputlisting[language=Kotlin]{flesh/listings/shm_check_permissions.kt}
	\label{list:shm_check_permissions}
\end{ListingEnv}\nobreak
\begin{ListingEnv}
	\caption{Запрос требуемых разрешений}
	\UseRawInputEncoding
	\footnotesize
	\lstinputlisting[language=Kotlin]{flesh/listings/shm_request_permissions.kt}
	\label{list:shm_request_permissions}
\end{ListingEnv}
\pagebreak
\begin{ListingEnv}
	\caption{Пример получения биометрических данных}
	\UseRawInputEncoding
	\footnotesize
	\lstinputlisting[language=Kotlin]{flesh/listings/shm_get_data.kt}
	\label{list:shm_get_data}
\end{ListingEnv}