\Intro

Геймификация спорта сегодня особенно актуальна: после введения ограничений в связи с распространением COVID-19 активность людей по всему миру начала неуклонно снижаться. В этот момент стали популярны самостоятельные тренировки в стиле ``несколько кругов по квартире'' или ``по лестнице подъезда туда-обратно''. Потребность в спорте остаётся актуальна и сейчас, когда ограничения постепенно снимаются. При этом, со временем, у многих людей, не привыкших к спорту, но чувствующих его необходимость в своей жизни, снижается уровень мотивации к занятиям разного рода активностями. 

Игра, будучи одной из форм деятельности человека, по определению является свободной развивающей деятельностью, выполняемой субъектом по желанию и ради удовольствия от процесса. Считается, что у такой формы деятельности есть четкий набор правил, а сама игра должна подготовить ребенка ко взрослой деятельности.

Однако в современном мире все меняется, и игра постепенно выходит на один уровень с остальными базовыми формами деятельности человека. При этом такая игра не готовит ко взрослым активностям, а представляет из себя комплиментарную сущность, которая добавляет то самое удовольствие от процесса, во время занятий другой формой деятельности. Такой подход к добавлению игровых элементов в некоторую деятельность, направленный на обучение деятельности или получение от неё большего удовлетворения как от процесса, принято называть геймификацией.  

Многие эксперты считают, что в современную эпоху упрощения и автоматизации геймификацию имеет смысл применять во всех сферах человеческой жизни, от медицины и до маркетинга.

Некоторое время назад, в рамках курса по гейм-дизайну в магистратуре ИММиКН, была рождена идея приложения, которое могло бы геймифицировать процесс беговых тренировок. После написания серьезного дизайнерского документа и концепции было принято решение воплотить эту идею в жизнь.

Так, данная работа посвящена разработке мобильной игры для геймификации бега с использованием средств геолокации на ОС Android. Приложение получило название ``Run, Listen, Run!''.

Отличительная особенность этого продукта заключается в том, что он позволяет геймифицировать уже имеющиеся у пользователя маршруты для бега, реагировать на состояние пульса и мотивировать к системности тренировок.

Далее в этой работе рассмотрим основную функциональность, методы геймификации бега и некоторые архитектурные решения, примененные нами в приложении ``Run, Listen, Run!''.


