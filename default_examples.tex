
% Если typeOfWork в SETUP.tex задан как 2 или 3, то начинать
% надо не с section (раздел), а с главы (chapter)
\section{Несколько примеров в~\LaTeX{}}
\label{sec:examples}

Некоторые часто используемые
команды приведены в качестве примера ниже (и варианты — в
комментариях). Мы рекомендуем внимательно прочесть данный
текст и изучить его исходный код прежде, чем начинать писать
свой собственный. Кроме того, можно дать и такой совет: идущий
ниже текст не убирать до самого конца, а просто оставлять его
позади своего собственного текста, чтобы в любой момент можно
было проконсультироваться с данными примерами.

\subsection{Как вставлять листинги и рисунки}

Для крупных листингов есть два способа. Первый красивый, но в нём не допускается
кириллица (у вас может встречаться в комментариях и
печатаемых сообщениях), он представлен на листинге~\ref{list:hwbeauty}.
\begin{ListingEnv}[H]% буква H означает Here, ставим здесь,
% элементы, которые нежелательно разрывать обычно не ставят
% посреди страницы: вместо H используется t (top, сверху страницы),
% или b (bottom) или p (page, на отдельной странице)
\begin{lstlisting}
#include <iostream>
using namespace std;

int main()
{
    cout << "Hello, world" << endl;
    system("pause");
    return 0;
}
\end{lstlisting}
%следующую команду для генерации подписи можно опустить,
% хотя рекомендуется все специальные элементы (таблицы, рисунки,
% листинги) подписывать. Если подпись пропустить, листинг также не получит
% номера и на него не сошлёшься в будущем
\caption{Программа “Hello, world” на \protect\cpp}
% далее метка для ссылки:
\label{list:hwbeauty}
\end{ListingEnv}

Второй не такой красивый, но без ограничений (см.~листинг~\ref{list:hwplain}).
\begin{ListingEnv}[H]
\begin{Verb}

#include <iostream>
using namespace std;

int main()
{
    cout << "Привет, мир" << endl;
}
\end{Verb}
\caption{Программа “Hello, world” без подсветки}
\label{list:hwplain}
\end{ListingEnv}

Можно использовать первый для вставки небольших фрагментов
внутри текста, а второй для вставки полного
кода в приложении, если таковое имеется.

Если нужно вставить совсем короткий пример кода (одна или две строки), то выделение  линейками и нумерация может смотреться чересчур громоздко. В таких случаях можно использовать окружения \texttt{lstlisting} или \texttt{Verb} без \texttt{ListingEnv}. Приведём такой пример с указанием языка программирования, отличного от заданного по умолчанию:
\begin{lstlisting}[language=Haskell]
fibs = 0 : 1 : zipWith (+) fibs (tail fibs)
\end{lstlisting}
Такое решение~--- со вставкой нумерованных листингов покрупнее
и вставок без выделения для маленьких фрагментов~--- выбрано,
например, в книге Эндрю Таненбаума и Тодда Остина по архитектуре
компьютера~\autocite{TanAus2013} (см.~рис.~\ref{fig:tan-aus}).

Наконец, для оформления идентификаторов внутри строк
(функция \lstinline{main} и тому подобное) используется
\texttt{lstinline} или, самое простое, моноширинный текст
(\texttt{\textbackslash texttt}).

\begin{figure}[p]% p означает, что нужно выделить для рисунка
% отдельную страницу; применяется для больших рисунков
\centering
%Здесь могла быть ваша лягушка.
\includegraphics[width=\textwidth]{img/tan-aus.png}
\caption{\label{fig:tan-aus}Пример оформления листингов в~\autocite{TanAus2013}}
\end{figure}

Использовать внешние файлы (например, рисунки) можно и на \href{http://overleaf.com}{overleaf.com}: ищите кнопочку upload.

\subsection{Как оформить таблицу}

Для таблиц обычно используются окружения table и tabular --- см. таблицу~\ref{tab:widgets}. Внутри окружения tabular используются специальные команды пакета booktabs — они очень красивые; самое главное: использование вертикальных линеек считается моветоном.

\begin{table}
\centering
\caption{\label{tab:widgets}Подпись к таблице --- сверху}
\begin{tabular}{llr}
\toprule
\multicolumn{2}{c}{Item} \\
\cmidrule(r){1-2}
Животное  & Описание    & Цена (\$) \\
\midrule
Gnat      & per gram    & 13.65      \\
          & each        & 0.01       \\
Gnu       & stuffed     & 92.50      \\
Emu       & stuffed     & 33.33      \\
Armadillo & frozen      & 8.99       \\
\bottomrule
\end{tabular}
\end{table}

\subsection{Как набирать формулы}

\LaTeX{} is great at typesetting mathematics. Let $X_1, X_2, \ldots, X_n$ be a sequence of independent and identically distributed random variables with $\text{E}[X_i] = \mu$ and $\text{Var}[X_i] = \sigma^2 < \infty$, and let
$$S_n = \frac{X_1 + X_2 + \cdots + X_n}{n}
      = \frac{1}{n}\sum_{i}^{n} X_i$$
denote their mean. Then as $n$ approaches infinity, the random variables $\sqrt{n}(S_n - \mu)$ converge in distribution to a normal $\mathcal{N}(0, \sigma^2)$.

\subsection{Как оформлять списки}

Нумерованные списки (окружение enumerate, команды item)…

\begin{enumerate}
  \item Like this,
  \item and like this.
\end{enumerate}

\dots маркированные списки \dots

\begin{itemize}
  \item Like this,
  \item and like this.
\end{itemize}

\dots списки-описания \dots

\begin{description}
  \item[Word] Definition
  \item[Concept] Explanation
  \item[Idea] Text
\end{description}

\Conc

Помните, что на все пункты списка литературы должны быть ссылки. \LaTeX\ просто не добавит информацию об издании из bib"/файла, если на это издание нет ссылки в тексте. Часто студенты используют в работе  электронные ресурсы: в этом нет ничего зазорного при одном условии: при каждом заимствовании следует ставить соответствующую ссылку. В качестве примера приведём ссылку на сайт нашего института~\autocite{mmcs}.

Для дальнейшего изучения \LaTeX\ рекомендуем книгу Львовского~\autocite{Lvo2003}: она хорошо написана, хотя и несколько устарела.
Обычно стоит искать подсказки на
\href{http://tex.stackexchange.com/}{tex.stackexchange.com}, а также
читать документацию по установленным пакетам с помощью
команды
\begin{Verb}
texdoc имя_пакета
\end{Verb}
или на \href{http://ctan.org/}{ctan.org}.

% Печать списка литературы (библиографии)
\printbibliography[%{}
    heading=bibintoc%
    %,title=Библиография % если хочется это слово
]
% Файл со списком литературы: biblio.bib
% Подробно по оформлению библиографии:
% см. документацию к пакету biblatex-gost
% http://ctan.mirrorcatalogs.com/macros/latex/exptl/biblatex-contrib/biblatex-gost/doc/biblatex-gost.pdf
% и огромное количество примеров там же:
% http://mirror.macomnet.net/pub/CTAN/macros/latex/contrib/biblatex-contrib/biblatex-gost/doc/biblatex-gost-examples.pdf